\documentclass[11pt]{article}
\usepackage[ngerman]{babel}
\usepackage[margin=1in,paper=a4paper]{geometry}
\usepackage{amsmath} 
\title{\textbf{Schaltbare Spiegel}\\ {Funktionswerkstoffe der Elektrotechnik, Sommersemester 2020}}
%\author{Dimitrios Maximilian Kalodikis (393188)}
\date{\today}
\begin{document}
\maketitle
\section{Abstract}
Heutzutage wird mit Hilfe moderner Technologie immer mehr auf Knopfdruck erreicht. Sogar Spiegel können, unter Beachtung der Gesetze der Optik und Verwendung von bestimmten Materialien und optischen Bauteilen (Yttrium, Lanthan, Prismen etc. ) in einen transparenten oder absorbierenden Zustand umgeschaltet werden und umgekehrt. Zunehmende Beachtung finden derartige Spiegel bei der Erforschung von Diffusion in Festkörpern sowie bei der effizienten Umsetzung von Licht- und Wärmeregulation in modernen Gebäuden.\\
Dabei gibt es verschiedene Ansätze für die Realisierung des genannten Verhaltens, die sich hinsichtlich Ansteuerung (Wasserstoffbegasung, elektrische Spannung, einwirkendes Licht etc.) aber auch den physikalischen und chemischen Effekten, die ausgenutzt werden (Elektrochromie, Nanopartikel), unterscheiden. Ziel der vorliegenden Arbeit ist es eine Übersicht über die Technologien und den Stand der Forschung zu bieten, wobei ein besonderer Fokus auf die verwendeten Materialien und Verarbeitung ihrer gerichtet ist.
\section{Arten von schaltbaren Spiegeln}
\subsection{Mechanisch schaltbare Spiegel}
Mechanisch schaltbare Spiegel sind in Form von Autorückspiegeln die verbreitetste Form von schaltbaren Spiegeln. Im Grunde genommen handelt es sich um handelsüblich Spiegel, d.h. rückseitig mit Aluminium (oder früher Silber) beschichtete Gläser, die allerdings prismatisch, will heißen keilförmig, geformt sind (Abbildung N). Somit wird der Effekt, dass auch an der Grenzfläche zwischen Glas und Luft ein teil der Strahlen reflektiert wird ausgenutzt, indem der Fahrer bei hellen Scheinwerfern den Spiegel kippt und statt der Reflexion am Aluminium die Reflexion an der Grenzfläche, welche vielfach schwächer – also dunkler – ist, sieht[1]. Das Aluminium darf dann aber nur ein dunkles Bild reflektieren, damit dieses nicht dominiert.

Diese Spiegel werden klassischerweise mittels Sputtering oder Bedampfen hergestellt.
\subsection{Elektrochrome Spiegel}
Der Effekt der Elektrochromie beruht auf einer reversiblen elektrochemischen Redox-Reaktion, wobei das Reduzieren und Oxidieren einer Elektrode ihre Bandlücke und somit ihre optischen Eigenschaften verändert [2]. Man kann nun also einen solchen Spiegel wie ein elektrochromes Glas aufbauen, d.h. zwei Glasplatten, die mit transparenten Elektroden beschichtet sind. Diese sind wiederum beschichtet: Beide Elektrode tragen eine Ionen enthaltende Schicht, wobei eine davon elektroaktiv ist (z.B. Wolframtrioxid). Diese umgeben den transparenten Elektrolyten [citation needed]. Das dem Betrachter abgewandte Glas kann rückseitig mit Aluminium beschichtet werden, sodass es nicht nur als Träger der Elektrode sondern auch des eigentlichen Spiegels fungiert. Alternativ ist auch eine spiegelnde Elektrode am rückwandigen Glas denkbar [4]. Durch Anlegen einer Spannung über die beiden Elektroden wird die Redox-Reaktion in Gang gesetzt und der Aufbau vor dem eigentlichen Spiegel verdunkelt.

1: https://www.leifiphysik.de/optik/lichtbrechung/ausblick/abblendspiegel\\
2: MONK, MORTIMER, ROSSEINSKY: ELECTROCHROMISM AND ELECTROCHROMIC DEVICES\\
4: US-Patent WO2004098953A2
\end{document}