%%%%%%%%%%%%%%%%%%%%%%%%%%%%%%%%%%%%%%%%%%%%%%%%%%%%%%%%%%%%%%%%%%%%%%%
% Thema: Poster
% Autor: Musterman
% Datum: 12.05
%%%%%%%%%%%%%%%%%%%%%%%%%%%%%%%%%%%%%%%%%%%%%%%%%%%%%%%%%%%%%%%%%%%%%%%
%
\documentclass[a0,portrait]{a0poster}
%\documentclass[portrait,a0,posterdraft]{a0poster}
%
%%%%%%%%%%%%%%%%%%%%%%%%%%%%%%%%%%%%%%%%%%%%%%%%%%%%%%%%%%%%%%%%%%%%%%%
% Paket zum Erzeugen der Spalten
\usepackage{multicol}

%%%%%%%%%%%%%%%%%%%%%%%%%%%%%%%%%%%%%%%%%%%%%%%%%%%%%%%%%%%%%%%%%%%%%%%
% Eingabekodierung
%\usepackage[ansinew]{inputenc}
\usepackage[utf8]{inputenc}
\usepackage[T1]{fontenc}
\usepackage{ae} %wenn Ulaute nicht gehen

\usepackage[ngerman]{babel}
%\begin{übler Trick um die Orinalüberschrift zu unterdrücken}
\addto{\captionsngerman}{\renewcommand*{\refname}{}} %Literaturüberschfit unterdrücken
%\ende{übler Trick}

%%%%%%%%%%%%%%%%%%%%%%%%%%%%%%%%%%%%%%%%%%%%%%%%%%%%%%%%%%%%%%%%%%%%%%%
% für Farbe
\usepackage{color} 
\definecolor{darkgreen}{rgb}{0,0.5,0} 
\definecolor{darkblue}{rgb}{0,0,0.5}

%%%%%%%%%%%%%%%%%%%%%%%%%%%%%%%%%%%%%%%%%%%%%%%%%%%%%%%%%%%%%%%%%%%%%%%
% Mathepaket
\usepackage{amsmath} 

%%%%%%%%%%%%%%%%%%%%%%%%%%%%%%%%%%%%%%%%%%%%%%%%%%%%%%%%%%%%%%%%%%%%%%%
% Ist für Boxen (wie das hier benutzte Ovalbox) zuständig
\usepackage{fancybox}

%%%%%%%%%%%%%%%%%%%%%%%%%%%%%%%%%%%%%%%%%%%%%%%%%%%%%%%%%%%%%%%%%%%%%%%
% Graphikpaket, ermöglicht png, jpg und pdf bilder
\usepackage{graphicx}

%%%%%%%%%%%%%%%%%%%%%%%%%%%%%%%%%%%%%%%%%%%%%%%%%%%%%%%%%%%%%%%%%%%%%%%
% Seiteneinstellungen
\renewcommand\baselinestretch{1.35}
\parskip=0.5\baselineskip

\parindent0mm %Einrücktiefe der ersten Zeile eines Absatzes
\topmargin-28pt
\marginparwidth0mm

%Ränder rechts/links
\oddsidemargin-13pt
\evensidemargin-13pt
\textwidth785mm
%\textheight1140mm

%%%%%%%%%%%%%%%%%%%%%%%%%%%%%%%%%%%%%%%%%%%%%%%%%%%%%%%%%%%%%%%%%%%%%%%
% Eigene Definitionen zur Erleichterung des Satzes
\newcommand{\spaltenbreite}{25}   % Spaltenbreite für Bilder
\newcommand{\bildbreite}{25cm}    % Einheitliche Bildbreite

%%%%%%%%%%%%%%%%%%%%%%%%%%%%%%%%%%%%%%%%%%%%%%%%%%%%%%%%%%%%%%%%%%%%%%%
% Box- und Spalteneinstellungen
\setlength{\fboxrule}{3.25mm} %Definiert die Linienstärke für nachfolgende fbox- und framebox-Befehle
\setlength{\fboxsep}{5mm} %Abstand zwischen Rahmen und Text bei den /fbox und /framebox Befehlen.
\setlength{\columnsep}{15mm}     %Spaltenabstand
\setlength{\columnseprule}{1pt}  %Balken zwischen Spalten {0pt}->keine Balken

%%%%%%%%%%%%%%%%%%%%%%%%%%%%%%%%%%%%%%%%%%%%%%%%%%%%%%%%%%%%%%%%%%%%%%%
% Einheitslänge für picture-Umgebungen
%\setlength{\unitlength}{1.0cm}
\unitlength1cm

%%%%%%%%%%%%%%%%%%%%%%%%%%%%%%%%%%%%%%%%%%%%%%%%%%%%%%%%%%%%%%%%%%%%%%%
% Grafikpfad, hier liegen alle Bilder
%\graphicspath{{images/}}

%%%%%%%%%%%%%%%%%%%%%%%%%%%%%%%%%%%%%%%%%%%%%%%%%%%%%%%%%%%%%%%%%%%%%%%
% Zaehler fuer lineale. Sie werden gebraucht, wenn das Linealmacro
% included wird
\newcounter{skalax}
\newcounter{skalay}


\begin{document}

%%%%%%%%%%%%%%%%%%%%%%%%%%%%%%%%%%%%%%%%%%%%%%%%%%%%%%%%%%%%%%%%%%%%%%%
% Kopf, hier funktioniert alles mit 'put'
%%%%%%%%%%%%%%%%%%%%%%%%%%%%%%%%%%%%%%%%%%%%%%%%%%%%%%%%%%%%%%%%%%%%%%%
%\parbox{\textwidth}
%{
  \begin{center}
    \vspace*{0.006\textheight}
   % Titel und Autor
    {\huge \textbf{Schaltbare Spiegel}\\Übersicht über Technologien\\}%[0.03\textheight]
    \vspace*{0.03\textheight}
    {\large \textbf{Gruppe 15: Nima Astan Varzeghan, Marc Alexander Depping, Andrés Felipe Fonseca Richter,\\Dimitrios Maximilian Kalodikis, Luka Lolovic, Juan Nicolas Pardo Martin}}
  \end{center}
  % Logos einbinden (per Hand plazieren)
  \begin{picture}(0,0)
    \put(70.0 , 4.6){\includegraphics[height=80mm]{logo}}
    \put(71.8 , 2.8){\LARGE \textsf{\textbf{TU Berlin}}}
    %\put(0.0  , 4.3){\includegraphics[height=83mm]{logo}}
    \put(0.0  , 8.1){\large \textsf{\textbf{Funktionswerkstoffe der Elektrotechnik}}}
    \put(0.0  , 6.75){\large \textsf{\textbf{Prof. Dr.-Ing. Ronald Plath}}}
    \put(0.0  , 5.4){\large \textsf{\textbf{Prof. Dr. rer. nat. Bernd Szyszka}}}
  \end{picture}
%}

%%%%%%%%%%%%%%%%%%%%%%%%%%%%%%%%%%%%%%%%%%%%%%%%%%%%%%%%%%%%%%%%%%%%%%%
% Einstellung des Eckenradius' der ovalen Boxen (\Ovalbox)
\cornersize*{5mm}


\linethickness{0.1mm}
\setlength{\fboxrule}{2.25mm}

%%%%%%%%%%%%%%%%%%%%%%%%%%%%%%%%%%%%%%%%%%%%%%%%%%%%%%%%%%%%%%%%%%%%%%% 
% neuer Kasten
\Ovalbox
{
  \parbox{\textwidth}{
     \begin{center} \textbf{\Large Zusammenfassung} \end{center}
Heutzutage wird mit Hilfe moderner Technologie immer mehr auf Knopfdruck erreicht. Sogar Spiegel können, unter Beachtung der Gesetze der Optik und Verwendung von bestimmten Materialien und optischen Bauteilen (Yttrium, Lanthan, Prismen etc. ) in einen transparenten oder absorbierenden Zustand umgeschaltet werden und umgekehrt. Zunehmende Beachtung finden derartige Spiegel bei der Erforschung von Diffusion in Festkörpern sowie bei der effizienten Umsetzung von Licht- und Wärmeregulation in modernen Gebäuden.\\
%Dabei gibt es verschiedene Ansätze für die Realisierung des genannten Verhaltens, die sich hinsichtlich Ansteuerung (Wasserstoffbegasung, elektrische Spannung, einwirkendes Licht etc.) aber auch den physikalischen und chemischen Effekten, die ausgenutzt werden (Elektrochromie, Nanopartikel), unterscheiden. Ziel der vorliegenden Arbeit ist es eine Übersicht über die Technologien und den Stand der Forschung zu bieten, wobei ein besonderer Fokus auf die verwendeten Materialien und Verarbeitung ihrer gerichtet ist.
  }
}

%%%%%%%%%%%%%%%%%%%%%%%%%%%%%%%%%%%%%%%%%%%%%%%%%%%%%%%%%%%%%%%%%%%%%%%
% neuer Kasten
\vspace*{0.002\textheight}
\Ovalbox
{
  \parbox{0.45\textwidth}{
    \begin{multicols}{2}
      
      \input{texta}
      \input{texta}
      
    \end{multicols}
  }
}

%%%%%%%%%%%%%%%%%%%%%%%%%%%%%%%%%%%%%%%%%%%%%%%%%%%%%%%%%%%%%%%%%%%%%%%
% neuer Kasten
%\vspace*{0.002\textheight}
\Ovalbox
{
  \parbox{0.45\textwidth}{
    \begin{multicols}{2}
      
      \input{texta}
      \input{texta}
      
    \end{multicols}
  }
}

%%%%%%%%%%%%%%%%%%%%%%%%%%%%%%%%%%%%%%%%%%%%%%%%%%%%%%%%%%%%%%%%%%%%%%%
% neuer Kasten
\vspace*{0.002\textheight}
\Ovalbox%\fbox
{
  \parbox{\textwidth}{
    \begin{multicols}{3}
      
      \input{textb}
      \input{texta}
      \input{literatur}
      
    \end{multicols}
  }
}

\end{document}
