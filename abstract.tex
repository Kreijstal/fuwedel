\begin{center} \textbf{\Large Zusammenfassung} \end{center}
Heutzutage wird mit Hilfe moderner Technologie immer mehr auf Knopfdruck erreicht. Sogar Spiegel können, unter Beachtung der Gesetze der Optik und Verwendung von bestimmten Materialien und optischen Bauteilen (Yttrium, Lanthan, Prismen etc. ) in einen transparenten oder absorbierenden Zustand umgeschaltet werden und umgekehrt. Zunehmende Beachtung finden derartige Spiegel bei der Erforschung von Diffusion in Festkörpern sowie bei der effizienten Umsetzung von Licht- und Wärmeregulation in modernen Gebäuden.\\
%Dabei gibt es verschiedene Ansätze für die Realisierung des genannten Verhaltens, die sich hinsichtlich Ansteuerung (Wasserstoffbegasung, elektrische Spannung, einwirkendes Licht etc.) aber auch den physikalischen und chemischen Effekten, die ausgenutzt werden (Elektrochromie, Nanopartikel), unterscheiden. Ziel der vorliegenden Arbeit ist es eine Übersicht über die Technologien und den Stand der Forschung zu bieten, wobei ein besonderer Fokus auf die verwendeten Materialien und Verarbeitung ihrer gerichtet ist.